\documentclass{sigplanconf}

% Email drafts to: M. George Hansen, Evan Laforge, Ketil Malde

\usepackage{amsmath}
\usepackage{amssymb}
\usepackage{natbib}
\usepackage{multirow}
\usepackage{setspace}
\usepackage{balance}
\usepackage[all]{xy}

\include{paper}
%include paper.fmt
%format (List (a) (b)) = "[" b "]_{" a "}"
%format Result_alpha = "\Varid{Result_\alpha}"
%format *> = "\mathbin{*\hspace{-5.8px}>}"
%format **> = "\mathbin{*\!*\hspace{-5.8px}>}"
%format ?> = "\mathbin{\raisebox{-.7px}{?}\hspace{-4px}>}"
%format ?== = "\mathbin{\raisebox{-.7px}{?}\hspace{-4px}\equiv}"
%format `replaceExtension` = "\backtick{replaceExtension}"
%subst string a = "\!\text{\sf ``" a "\char34}\!"
%format context = "\Keyword{context}"
%format dependsS = "\Varid{depends\!^{*}}"


\newcommand{\file}{\textsf}
\newcommand{\prog}{\texttt}
\newcommand{\make}{\prog{make}}

\begin{document}
\conferenceinfo{ICFP'12,} {September 27--29, 2010, Baltimore, Maryland, USA.}
\CopyrightYear{2012}
\copyrightdata{978-1-60558-794-3/10/09}

\preprintfooter{}   % 'preprint' option specified.

\title{Shake -- A Better Make}
% \subtitle{}

\authorinfo{Neil Mitchell}
           {\verb"ndmitchell@gmail.com"}

\maketitle

\begin{comment}
1. State the problem
2. Say why it�s an interesting problem
3. Say what your solution achieves
4. Say what follows from your solution
\end{comment}

\begin{abstract}
Build tools, such as \make{}, are typically used for compiling source code into executables. Unfortunately, most build tools fail to deal properly with generated source files, especially when the dependencies of those generated files can only be determined after the files have been generated. Generated files are common in many large software systems, resulting in awkward hacks to paper over build tool inadequacies. We describe how to generalise build tools to deal properly with generated files. We have implemented our ideas as the Haskell library Shake, which has been used by Standard Chartered for the last three years as the basis for a complex build system involving millions of lines of code in many languages.
\end{abstract}

\category{D.3}{Software}{Programming Languages}

\terms
Languages

\keywords
build-system, compilation, Haskell

\section{Introduction}
\label{sec:introduction}

A build tool, such as \make{}, takes a set of input files, plus some rules, and produces some output files. Taking \make{} as an example, a build rule looks like:

\begin{code}
{-"\textsf{result.tar : a.txt b.txt}"-}
    {-"\textsf{tar -cf result.tar a.txt b.txt}"-}
\end{code}

\noindent This rule builds the tar archive \file{result.tar} from the inputs \file{a.txt} and \file{b.txt}. Whenever \file{a.txt} or \file{b.txt} changes \file{result.tar} will be regenerated.

What if we want to build \file{result.tar} from the list of files in \file{list.txt}? \make{} offers no easy way to express this pattern (there are workarounds, discussed in \S\ref{sec:make_hacks}, but none are pleasant or effective). However, using the build tool we develop in this paper, we can write:

\begin{code}
"result.tar" *> \_ -> do
    need ["list.txt"]
    contents <- readFileLines "list.txt"
    need contents
    system' "tar" $ ["-cf","result.tar"] ++ contents
\end{code}

This rule describes how to build \file{result.tar}. We depend on (|need|) the file \file{list.txt}. We read \file{list.txt}, and store each line in |contents| -- which is a list of the files that should go into \file{result.tar}. Next, we depend on all the files in |contents|, and finally call the \prog{tar} program specifying the files in |contents|. If \file{list.txt} changes, or any of the files listed by \file{list.txt} change, then \file{result.tar} will be rebuilt.

The key difference from \make{} (and nearly all other build tools) is that rather than specifying the dependencies of a rule \textit{in advance}, we allow further dependencies to be specified \textit{after} using previous dependencies. This difference is crucial to properly handle generated files.

We have implemented our build tool as a Haskell library, named Shake, which is available online\footnote{\url{http://hackage.haskell.org/package/shake}}. Shake properly handles generated files and includes the important features of \make{}, such as minimal rebuilds (running only a subset of the rules when some subset of the inputs change), and parallelising the build (running multiple independent rules at the same time). By implementing Shake as a Haskell library we allow rules to be written using the full power of Haskell, including the use of modules and functions to properly structure large systems.

\subsection{Contributions}

\begin{itemize}
\item We describe the theory underlying \make{} (\S\ref{sec:theory_make}), and how to revise this theory to properly handle generated files (\S\ref{sec:theory_shake}).
\item We describe how to extend our theory with a cache, to enable minimal rebuilds (\S\ref{sec:theory_shake_cache}).
\item We describe how to implement our build tool in Haskell, including how to present the underlying theory in a way that is practically usable (\S\ref{sec:user_view}), and how to implement it efficiently (\S\ref{sec:developer_view}). While Haskell is not essential to implement our build tool, it offers a number of advantages -- primarily making IO effects explicit and providing nice syntactic sugar.
\item We describe how additional features can be added to our build tool, including a Lint tool (\S?), a profiling tool (\S?) and a dependency analysis tool (\S?).
\item We allow rule dependencies and targets to be arbitrary values. While we can represent files (\S?), we are also able to represent static configuration information (\S?), the list of files in a directory (\S?) and commands that produce multiple results (\S?).
\item We have implemented a large build system using Shake (\S?). The build system is 9000 of lines of Haskell, building over a million lines of source code and over a million lines of generated code, written in many programming languages. We originally implemented this build system using \make{}, but the result was slow to run, hard to maintain, and frequently caused spurious compile failures. Since using Shake, our build system has ceased to be a problem.
\item We give a number of guidelines to follow when using Shake, based on our experiences (\S?).
\end{itemize}


\section{Theory}
\label{sec:theory}

In this section we describe the theory that underpins of \make{}, then the underlying theory of our build system, named Shake. We then describe how to support minimal rebuilds in both theories. In \S\ref{sec:user_view} and \S\ref{sec:developer_view} we show how to implement these ideas in a usable tool.

\subsection{Theory of Make}
\label{sec:theory_make}

While the \make{} tool is heavily file based, that is not an essential property of the ideas behind \make{}. We use the type |Key| for things that can be created or are dependencies (i.e. file names) and the type |Value| for the values associated with a |Key| (i.e.  file contents). Using this abstraction, we can model \make{} as:

\begin{code}
data  Rule =
      {  creates  :: Key
      ,  depends  :: List alpha Key
      ,  action   :: List alpha Value -> Value
      }

make :: Set Rule -> Key -> Value
\end{code}

The |make| function takes a (possibly infinite) set of rules and the target |Key| to build, and returns the |Value| associated with that |Key|. A rule can be modelled as the |Key| it |creates|, the |Key|s it |depends| on, and the |action| that takes the depended upon |Value|s and produces the result |Value|. We use |List alpha bullet| to denote a list which must have length |alpha|, requiring that the |depends| list and the list passed to |action| are equal in length.

We restrict our model to only building one target, while \make{} allows multiple targets (i.e. a list of |Key|s to build). However, we can encode multiple targets by creating a distinguished rule that depends on all the targets and returns all their values, and then make that rule the single target.

\subsubsection{Correctness}

We say a |Rule| is associated with a |Key| if it has that |Key| as its |creates| field. Given a |Rule| |r| we can define |dependsS r| as all the |Key|s this rule requires to run, including the dependencies of its dependencies. We can compute |dependsS r| as the union of |depends r| and |dependsS| on all the rules associated to |depends r|.

For a call to |make| to be valid, given a set of rules and a target, we require:

\begin{enumerate}
\item \textit{No duplicate rules} - no |Key| may have two associated rules.
\item \textit{No missing rules} - every item in a |depends| list must have an associated rule.
\item \textit{Target rule} - there must be a rule associated with the target.
\item \textit{Finite} - for all rules, |dependsS r| must be finite.
\item \textit{Acyclic} - for all rules, |dependsS r| must not contain |creates r|.
\end{enumerate}

\noindent Given a valid call to |make|, there exists a finite list of |action|s that when run in order produces the result. Provided each |action| terminates, the result can be obtained.

\subsection{Theory of Shake}
\label{sec:theory_shake}

Using the same terminology from the previous section, we can model our build tool as:

\begin{code}
data  Rule =
      {  creates  :: Key
      ,  action   :: Action
      }

data Action  =  Depends Key (Value -> Action)
             |  Result Value

shake :: Set Rule -> Key -> Value
\end{code}

The |shake| function takes a (possibly infinite) set of rules and the target |Key| to build, and returns the |Value| associated with that |Key|. A rule can be modelled as the |Key| it |creates|, and the |action| that creates the result. The |Action| either returns the |Result| |Value|, or requires a new dependency with |Depends| -- specifying the |Key| it depends on, plus a function that takes the |Value| of that |Key| and provides a new |Action|.

The big difference from |make| is the introduction of dynamic dependencies. A rule can dynamically, based on the values of previous dependencies, require additional dependencies. We can easily translate a |make| |Rule| to a |shake| |Rule|, but the reverse is not true -- |shake| is strictly more powerful than |make|.

\subsubsection{Correctness}

A call to |shake| is correct if the |Value| associated with the target |Key| can be built. Unlike |make|, we cannot check this property before running any |action|s, since new dependencies can be added by running an |action|. Like |make|, we require that each |Key| is associated with exactly one |Rule|. Given a function which finds the rule associated with a key (|find|), we can run a build system as:

\begin{code}
shake rules target = fromKey target
    where
        fromKey k = fromAction (action (find rules k))

        fromAction (Result val) = val
        fromAction (Depends dep act) =
            fromAction (act (fromKey dep))
\end{code}

To reduce a |Key|, we find its |Rule| and reduce its |Action|. To reduce an |Action|, if it is a |Result| we return the value, if it is a |Depends| we reduce the dependency then reduce the result of the action. Note that in this simple implementation every time a |Key| is required we compute its |Value| from scratch -- an action may be run many times.

A call to |shake| is correct if it terminates. We can guarantee termination if:

\begin{enumerate}
\item We require that every |action| function terminates.
\item To ensure that the recursion in |fromAction| terminates, we require that |Rule| produces a finite number of |Depends| constructors before producing a |Result|.
\item To ensure that the recursion in |fromKey| terminates, we require that no |Rule| depends on its |create| value, either directly or through a sequence of other rules.
\item To ensure that the mutual recursion between |fromKey| and |fromAction| terminates, we require that there are a finite number of |Key| values required to build the |target|.
\end{enumerate}

%
% \begin{itemize}
% \item \textit{Result} rules are those whose |action| is a |Result|.
% \item \textit{Reducible} rules are those which are not \textit{results}, but where the immediate dependencies are all \textit{results}.
% \item \textit{Required} rules are those which are not \textit{results}, but which create something in the set of output keys, or who create something that a \textit{required} rule has in it's |Defer| list. We can guarantee that all \textit{required} rules must be evaluated for the build system to finish.
% \end{itemize}
%
% Like in \make{}, the rules |creates| must all be distinct.
%
% Given a set of rules, we are finished when all the \textit{required} rules are \textit{results}. We can proceed one step by taking a rule that is both \textit{required} and \textit{reducible} and running the deferred function on it (\textit{reducing} it). If there are no rules that are both \textit{required} and \textit{reducible} then we are stuck, and the build system has a logical inconsistency, either because there is no rule to build something, or the build rules form a cycle.
%
% Note that we cannot know if the build system is consistent until it has completed. In particular, there are two ways for a build system to loop. Most obviously, an \textit{required} and \textit{reducible} rule could have a deferred action that never terminates. Alternatively, such a rule could immediately produce the same dependencies as previously, without making progress. We assume that each rule is structured in a way that there are a finite number of steps, and each step takes a finite amount of time (in practice, it is hard to write a rule that does not obey this property).
%

\subsection{Minimal Rebuilds in Make}
\label{sec:cached_make}

One of the important properties of \make{} is minimal rebuilds. In any execution of |make|, each |action| will be run at most once. If the last time the action was run it had the same inputs, the action will not be repeated. The \make{} tool uses file modification times, and assumes a target is valid if all its dependencies have older modification times. We can model this behaviour with a function that looks up the modification times:

\begin{code}
history :: Key -> Maybe Value
\end{code}

If we restrict ourselves to |Key|s which are files, and |Value|s which are modification times (as \make{} does), then we can write a function |history| which returns |Just| the modification time, or |Nothing| to indicate the file does not exist (and thus has never been built). An |Rule|s |action| can be skipped if the |creates| file exists, and all the |depends| times are older. Before running a rules action, all its dependencies must already have been created, and thus history must always return |Just|. We can check if a Rule is valid with respect to the history with:

\begin{code}
validHistory :: Rule -> Bool
validHistory r = case history (creates r) of
    Nothing -> False
    Just t -> all ((< t) . fromJust . history) (depends r)
\end{code}

This scheme is limited to files, for two reasons:

\begin{enumerate}
\item Our |history| function relies on being able to access the history for a previous run which is stored on disk. We can eliminate this problem by storing the information in a database as we are executing, and then reloading it from disk on each execution.
\item Our |validHistory| function relies on a |Value| which is monotonically increasing (e.g. time). We can eliminate this problem if whenever we complete an action we give it a real timestamp.
\end{enumerate}

We can make these changes to produce:

\begin{code}
data  Info = Info
      {  value :: Value
      ,  built :: Time
      }

history :: Key -> Maybe Info

validHistory r = case history (creates r) of
    Nothing -> False
    Just i -> all ((< built i) . built . fromJust . history) (depends r)
\end{code}

However, this scheme is incorrect for |Value|s that are stored by both the file system \textit{and} in our database. Consider a file, which has a file system modification time, and also has a |Value| in our database. If the file is edited externally, then the |Value| in our database will be incorrect. We can solve this problem by also checking that, where the |Value| is stored externally, it is still consistent:

\begin{code}
data Stored = NeverStored | NotStored | Stored Value

stored :: Key -> Stored

validStored :: Key -> Value -> Bool
validStored k v = case stored k of
    NeverStored -> True
    NotStored -> False
    Stored value -> v == value
\end{code}

When attempting to read a value stored externally there are three possible states. Either the value is never stored, in which case it exists only in our database, and is always consistent. Or the value is sometimes stored, but not currently available externally -- such as a file that does not exist on disk, in which case the action must be rerun. Or the value is stored, and has an external value -- in which case the value is valid if it matches what we have stored. We can avoid running the action associated with a rule if it is valid with respect to the stored values, and with respect to the history.

In order to remain consistent throughout the execution, we require that all |stored| values do \textit{not} change during the execution, other than as a result of running an |action| which |creates| it. The \make{} tool requires a similar property or it becomes inconsistent.

While the timestamp approach works reasonably for \make{}, it can fail if the system clock changes, or if a file is modified but has its value set to an older timestamp -- such as when extracting a backup which resets the timestamp. Instead of relying on the system time, we can instead store a counter in our database, and increment it every time we require a time value, thus guaranteeing that our time is correctly ordered.

After running an action, we typically store a new time in the |history|. However, if the result of the |action| has not changed we \textit{do not} update the time. This allows any files that depend on this file to avoid recalculation, which can significantly reduce the number of actions that need to be run.

\subsection{Minimal Rebuilds in Shake}
\label{sec:cached_shake}

Our original algorithm calculated every file repeatedly. We can easily add a cache mapping |Key| to |Value| and avoid reducing a rule that has already been reduced in this execution.

To obtain minimal rebuilds across executions we can use the same approach as described in the previous section. However, we cannot determine whether a |history| is valid based on the rule, as we do not have the |depends| available, and do not want to rerun the |action| just to get the dependencies. Therefore, we just store the dependencies alongside.

\begin{code}
data  Info = Info
      {  value :: Value
      ,  built :: Time
      ,  depends :: [Key]
      }

history :: Key -> Maybe Info

validHistory r = case history (creates r) of
    Nothing -> False
    Just i -> all ((< built i) . built . fromJust . history) (depends i)
\end{code}

We can reuse the |validStored| unmodified.

\subsubsection{Unchanging Files}
\label{sec:unchanging_files}

Now let us consider the case where a rule runs, but the result is the same as last time. As an example, consider a generated source file. If the generator changes it is necessary to regenerate the file, but there is a chance the result will be the same. In \make{}, the solution is to disallow such changes, and always \prog{touch} the result after a rule completes. However, we can do better, avoiding unnecessary computation.

Instead of storing just the built time, we can also store the |changed| time. Whenever we build a file we update its |built| time, but if the value is the same as last time, we leave its |changed| time the same. When checking |validHistory| we can replace the condition as:

\begin{code}
all ((< built i) . changed . fromJust . history) (dependedUpon i)
\end{code}

By introducing two timestamps for a single |Key|, we can reduce the number of builds required. For certain practical examples, this improvement can reduce rebuild times after a change from many minutes to seconds.

\section{Shake in Haskell}
\label{sec:user_view}

In order to make the theory from \S\ref{sec:theory} practically usable as a Haskell library, we need to make a number of design decisions. In particular, we describe how to replace |Key| and |Value| with standard polymorphic values, how to integrate IO into the |action| function and how to define an infinite set of rules. We first present the interface to Shake, then in \S\ref{sec:developer_view} we describe how to implement the internals of the library.

\subsection{A Shake Example}

\begin{figure}
\begin{code}
import Development.Shake
import System.FilePath

main = shake def $ do
    want ["Main"]

    "Main" *> exe -> do
        cs <- getDirectoryFiles "." "*.c"
        let os = map (`replaceExtension` "o") cs
        need os
        system' "gcc" $ ["-o",exe] ++ os

    "*.o" *> \o -> do
        let c = replaceExtension o "c"
        need [c]
        headers <- liftIO (cIncludes c)
        need headers
        system' "gcc" ["-o",o,"-c",c]
\end{code}
\caption{Demo build system in Shake.}
\label{fig:demo}
\end{figure}

We give an example Shake program in Figure \ref{fig:demo}. Running this program will build \file{Main} from all the \file{*.c} files in the current directory. If we add or remove a \file{.c} file, or change any of the \file{.c} files or the header files they @#include@, then the necessary files will be recompiled.

The script produces (|want|'s) the file \file{Main}. To generate \file{Main} we list all the \file{*.c} files in the current directory, change their extensions to \file{*.o}, require those files to be built (|need| them), then call \prog{gcc} to link them. To build any \file{*.o} file we take the associated \file{*.c}, make sure it's been built, then call the function |cIncludes| to get all headers it includes (|cIncludes| can be defined in terms of \prog{gcc -M}). We require those headers, then we call \prog{gcc} to do the compilation.

This script demonstrates a number of features of Shake based build systems:

\begin{itemize}
\item It's a full Haskell program with a |main| entry point. While |main| can simply call |shake|, it can also do anything it likes, such as command line processing (see \S?).
\item The call to |getDirectoryFiles| is tracked (see \S?), if the results change, it will trigger a rebuild.
\item We run arbitrary system commands in the internals.
\item We fully track the dynamic dependencies introduced by header files. The function |cIncludes| given a source file returns all the files required by transitive @#include@ directives. (In \S? we show a better way to define transitive dependencies.)
\end{itemize}

\subsection{Core Shake}

\begin{figure}
\begin{code}
data  ShakeOptions = ShakeOptions
      {  shakeFiles :: FilePath
      ,  shakeParallel :: Int
      ,  ellipses
      }
    deriving (Default)

data Rules a
    deriving (Monad, Monoid)

data Action a
    deriving (Monad, LiftIO)

class (
    Show key, Show value,
    Typeable key, Typeable value,
    Hashable key, Hashable value,
    Eq key, Eq value,
    Binary key, Binary value,
    NFData key, NFData value
    ) => Rule key value | key -> value where
    validStored :: key -> value -> IO Bool
    validStored k v = return True

run :: ShakeOptions -> Rules () -> IO ()

action :: Make a -> Rules ()

rule, defaultRule :: Rule key value =>
    (key -> Maybe (Make value)) -> Rules ()

apply :: Rule key value => List alpha key -> Make (List alpha value)
apply1 :: Rule key value => key -> Make value
\end{code}
\caption{Primitive operations in Shake}
\label{fig:shake_core}
\end{figure}

The primitive interface to Shake is given in Figure \ref{fig:shake_core} -- everything else is defined on top. At its essence, a Shake build system is a set of rules. We can run the rules with |run|, create new rules with |rule|/|defaultRule|/|action|, and when defining a rule we can express a dependency with |apply|/|apply1|.

We execute a set of rules with the |run| function, which also takes an options record. Typical options include which file to store the cached versions in (|shakeFiles|), and the number of processors to use (|shakeParallel|). These options are also used to select which mode to run Shake in, as described in \S\ref{sec:tools}.

Every rule that is defined or applied in Shake must be a member of the |Rule| class. The |Rule| class defines the method |validStored| to determine whether a value is consistent with the value stored externally. We require that each rule |key| has only one type of |value|, this restriction is not technically required, but forces users to keep their rules simple. Each |key|/|value| type must also be in several type classes:

\begin{description}
\item [Typeable] We allow multiple types of rules in a single build system. To distinguish the types, and to match on only one type at a time, we require a |Typeable| constraint, allowing us to obtain an explicit |TypeRep| \cite{typeable}.
\item [Eq] We require equality to match the values.
\item [Binary] We require |Binary| so we can store the values in a database, to achieve minimal rebuilds.
\item [Hashable] We require |Hashable| to perform fast searching.
\item [Show] We require |Show| for debugging messages and logging.
\item [NFData] We require |NFData| to ensure that values are fully evaluated when they are computed, ensuring that errors happen at the appropriate place.
\end{description}

Standard Shake rules are defined with |rule|, which requires a function which takes a single |key| value, and returns |Nothing| to indicate that this rule does not build this |key|, or |Just| with the steps necessary to build the associated |value|. If two rules match the same key then an error is raised. The function |defaultRule| allows a rule to be defined with a lower priority, which is used if no standard rules match -- for an example see \S\ref{sec:user_view_files}. Instead of defining targets to build, we allow rules with no target, which are always run. The |Rules| type is a |Monoid|, allowing two sets of rules to be appended to produce a new set of |Rules|. In practice, the syntactic sugar supported by |Monad| is often a more natural way to define rules in a build system, so we also make |Rules| a |Monad|. The |Action| type is a |Monad| and has an instance for |MonadIO|, allowing users to call arbitrary |IO| functions using |liftIO| to translate |IO alpha| to |Action alpha|.

The |apply1| function takes a |key|, builds it, and returns the associated |value|. The |apply1| function finds a rule of the appropriate type whose application returns |Just|, then runs the action. The |apply| function can be thought of as |mapM apply1|, but is defined to build all necessary dependencies in parallel (see \S\ref{sec:parallelism}).

We have deliberately kept the core of Shake small and extensible, supporting all instantiations of |Key|/|Value| outside, thus focusing purely on the building side. We describe how to implement this core in \S\ref{sec:developer_view}.

\subsection{Defining Rule Types}

To work correctly, the |key|/|value| types of |apply1| and |rule| must match each other. To aid end users, we suggest that most rule authors define sugar functions, as we have done for the rule types included with Shake. The simplest rule available is one that determines if a file exists, similar to the standard function |System.Directory.doesFileExist|, but where the dependency is tracked. As an example, consider the following rule:

\begin{code}
".config" *> \dest -> do
    b <- doesFileExist ".config.user"
    let src = if b then ".config.user" else ".config.default"
    copyFile src dest
\end{code}

This rule creates \file{.config} by copying either \file{.config.user} if it exists, or otherwise copying \file{.config.default}. The intention is that there is a default configuration, but it can be overridden by then user. Note the use of dynamic dependencies -- if the user has a \file{.config.user}, and the \file{.config.default} changes then the rule will not be run, however if the file didn't exist it would be.

\begin{figure}
\begin{code}
import System.Directory as IO

newtype Exist = Exist FilePath
-- plus instances for |Typeable|, |Show|, |Eq|, |Hashable| and |Binary|

instance Rule Exist Bool where
    validStored (Exist x) b = liftM (== b) $ IO.doesFileExist x

doesFileExist :: FilePath -> Action Bool
doesFileExist = apply1 . Exist

defaultRuleDoesFileExist :: Rules ()
defaultRuleDoesFileExist =
    defaultRule $ \(Exist x) -> Just $
        liftIO $ IO.doesFileExist x
\end{code}
\caption{Implementation of |doesFileExist|.}
\label{fig:doesfileexist}
\end{figure}

We define |doesFileExist| in Figure \ref{fig:doesfileexist}. We first define a |newtype| wrapper to represent the |key| type, deriving all the necessary classes. We define an instance for |Rule Exist Bool|, where |validStored| simply checks that the existence of the file matches. We define |doesFileExist| as very gentle sugar around |apply1|, the primary purpose is to tie down the types, so users cannot make mistakes. Finally we define a |defaultRule| which runs the action. Anyone wishing to use |doesFileExist| should include |defaultRuleDoesFileExist| in their rule set.

We use a restricted module export list to export only |doesFileExist| and |defaultRuleDoesFileExist|, and thus we are guaranteed that the types match. Note that if a value changes we are likely to call |IO.doesFileExist| twice in a row, once to invalidate the old value in |validStored|, and once to compute the new value in |default_|. We can alleviate this problem with a single element cache -- although in practice we find it not to be a problem, since the operating system typically caches all file queries.

We end with a note of caution. We require that any stored value remains consistent other than when a rule actively changes in. Since this rule does not force the existence of a file, it should not be used on any files that may be generated by the build system -- |doesFileExist| should return the same at the start and end of the compilation.

\subsection{File Based Rules}

\begin{figure}
\begin{code}
import System.Directory as IO

newtype File = File FilePath
newtype FileTime = FileTime Int
-- plus all necessary instances

getFileTime :: FilePath -> IO (Maybe FileTime)
getFileTime x = do
    b <- IO.doesFileExist x
    if not b then return Nothing else do
        TOD t _ <- IO.getModificationTime x
        return $ Just $ FileTime $ fromIntegral t

instance Rule File FileTime where
    validStored (File x) t = fmap (== Just t) $ getFileTime x

need :: [FilePath] -> Action ()
need xs = (apply $ map File xs :: Action [FileTime]) >> return ()

defaultRuleFile :: Rules ()
defaultRuleFile = defaultRule $ \(File x) -> Just $ do
    res <- liftIO $ getFileTime x
    let msg = "Error, file does not exist and no available rule: " ++ x
    return $ fromMaybe (error msg) res

want :: [FilePath] -> Rules ()
want xs = action $ need xs

(?>) :: (FilePath -> Bool) -> (FilePath -> Action ()) -> Rules ()
(?>) test act = rule $ \(File x) ->
    if not $ test x then Nothing else Just $ do
        liftIO $ createDirectoryIfMissing True $ takeDirectory x
        act x
        res <- liftIO $ getFileTime x
        let msg = "Error, rule failed to build the file: " ++ x
        return $ fromMaybe (error msg) res

(**>) :: [FilePattern] -> (FilePath -> Action ()) -> Rules ()
(**>) test act = (\x -> any (x ?==) test) ?> act

(*>) :: FilePattern -> (FilePath -> Action ()) -> Rules ()
(*>) test act = (test ?==) ?> act
\end{code}
\caption{Implement of file rules.}
\label{fig:file_rules}
\end{figure}

While our build system is not restricted to rules dealing with files, in practice many build systems are file orientated. For files, the filename is a sensible |Key|, but |Value| could be either the on-disk modification time or a hash of the file contents (e.g. SHA1). We found in that modification time is faster (significantly faster for large files) and being able to rebuild something with \prog{touch} is highly convenient! Of course, our design allows anyone to define a new file type, based on file hashes. Using modification time for values, we define the functions for working with files in Figure \ref{fig:file_rules}.

To force files to be built, we define |need| and |want|. The |need| action demands all the modification times associated with the filenames, then throws away the result -- a computation depends on the time a file was built, but rarely uses the modification time in its computation. We define |want| as simply a |need| run as an action.

We define |defaultRuleFile| as a rule that simply checks if the file is already present, and if so uses it. Source files will have no associated rules, so this rule just records their modification time. If a file has no rules (since any rules would be run in preference to the default rule), and does not exist, then we give an error to the user. As with |defaultRuleDoesFileExist|, anyone using |want|/|need| should include |defaultRuleFile| in their rule set.

We define new file rules with |(?>)|, which takes a predicate to match against the file name and an action to run. This function forces the correct types, and obtains the modification time afterwards. Before running the action we create the directory containing the output file, an idea taken from the Ninja build system \cite{ninja}. We found that when running a large set of newly written rules, often one rule would create the output directory while another did not -- meaning some rule orderings worked while others failed. Automatically creating the output directory removes this source of failure.

While |(?>)| is the ultimate file creation rule, we define two additional operators, in terms of a wildcard match operator |(?==)|. We define |(*>)| for matching a single pattern, for example |"*.c" *> ellipses|, in a very similar manner to \make{}. We also define |(**>)| for matching any one of a set of patterns. In our large evaluation, we found that |(?>)| was used about 5\% of the time, |(**>)| was used about 10\% and |(*>)| was used about 85\% of the time. In our experience, use of |(?>)| is a sign that the build system is poorly designed and should be rethought (our large evaluation used |(?>)| exclusively for compatibility with the system it replaced).

As a practical concern, when writing file rules, often there are multiple types of object that build \file{*.o} files. This problem can be easily solved by making Haskell produce \file{*.hs.o}, C produce \file{*.c.o} etc.


\subsection{Automatically Include Default Rules}
\label{sec:include_default_rules}

With the rule types already defined, users can write a build system using Shake. Unfortunately, if the user forgets to include the rule |defaultRuleFile|, there is likely to be a runtime error. Instead of requiring the user to remember to include the default rules, we instead define a wrapper function |shake| which includes all default rules available:

\begin{code}
shake opts rules = run opts $ do
    defaultRuleDoesFileExist
    defaultRuleFile
    ellipses
    rules
\end{code}

The astute reader may be wondering why we can't move |defaultRule| into the |Rule| type class, and avoid requiring such a wrapper. Alas, that solution doesn't work for reasons described in \S\ref{sec:dynamically_typed}, because we always need to have an explicit rules of each type to deserialise dynamically typed values.

\subsection{Additional Functions}

Using the file rules from the previous section, we can define new versions of operations such as |readFile|/|copyFile|, which automatically call |need|:

\begin{code}
readFile' :: FilePath -> Action String
readFile' x = need [x] >> liftIO (readFile x)

copyFile' :: FilePath -> FilePath -> Action ()
copyFile' old new = need [old] >> liftIO (copyFile old new)
\end{code}

These operations mirror their standard counterparts, but automatically include a |need| call. It is never harmful to include an additional call to |need|, in fact, sometimes it is beneficial! Consider the following example:

\begin{code}
"file.txt" *> \out ->
    need [out ++ ".part1", out ++ ".part2"]
    part1 <- readFile' (out ++ ".part1")
    part2 <- readFile' (out ++ ".part2")
    writeFile' out (part1++part2)
\end{code}

Here we use the additional function |writeFile'|, simply |writeFile| lifted to the |Action| monad. This rule would work correctly without the |need| call on the second line, as both |readFile'| calls include |need| within them. However, when running the build system in parallel, the additional |need| allows both dependencies to be computed at the same time, while the |need| calls inside |readFile'| serialise building of the dependencies.

We define a number of additional functions for working with files. The functions |readFileLines|/|writeFileLines| deal with files which should be treated as a list of lines. The function |writeFileEq| writes a file, but only if the contents have changed, and is useful for avoiding redundant computation due to unchanging files (see \S\ref{sec:unchanging_files}).

We define the |system'| function which take a list of command line arguments, and escapes them as necessary, before calling the underlying |system| command. This function also contributes generates profiling information (see \S\ref{sec:profiling}). The |system'| function should be used carefully, as it cannot tell which files the system command may depend on, so explicit |need| commands must be used (see \S\ref{sec:lint} for ways to detect such a problem). We recommend writing wrappers around system commands which insert the appropriate |need| calls.


\section{Implementing Shake}
\label{sec:developer_view}

We have implemented Shake, and used it extensively. In this section we sketch some of the central implementation challenges, and how they can be overcome. Readers interested in the full implementation are welcome to download the associated package (see \S\ref{sec:introduction} for a link). In the implementation of Shake there are several goals:

\begin{description}
\item[Correctness] -- we always execute enough rules to obtain correctness, but we never rebuild something that could be reused.
\item[Efficiency] -- we aim to make the implementation efficient, while remembering that often executing a rule is expensive.
\item[Parallelism] -- we try and do lots of work in parallel where possible.
\item[Error feedback] -- if the build system fails for some reason, such as the rules being incorrect as per \S\ref{sec:theory_shake}, then it's nice to give useful error messages. While this may sometimes conflict with efficiency (maintaining extra state to only be used in an error), it is necessary to make Shake usable.
\end{description}

\subsection{Dynamically Typed Values}
\label{sec:dynamically_typed}

A single Shake program can use multiple types of keys and values. To work with heterogenous values in Haskell we define:

\begin{code}
data Any = Any (forall alpha bullet
    (Typeable alpha, Eq alpha, Binary alpha, Hashable alpha, Show alpha) => alpha)
\end{code}

This definition, using existentials \cite{existentials}, allows anything supporting all the type classes required to be stored with type |Any|. We can define |Key| and |Value| types in terms of |Any|. We can implement |Eq|, |Show| and |Hashable| instances for |Any| without difficulty, often by appealing to the |TypeRep| provided by the |Typeable| instance.

Implementing a |Binary| instance for |Any| is harder. Serialising a value is easy, but deserialising a value is problematic -- we can deserialise a |TypeRep|, but then we need to obtain the |get| method from the |Binary| instance associated with that type. The solution is to keep a table mapping |TypeRep|s to |Any| values containing the appropriate |Binary| instance. This table must include every type stored in the file we are reading, or it cannot be deserialised. We populate the table using all rules defined by |rule| and |defaultRule| in |Rules| -- which means we cannot move |defaultRule| into the |Rule| type class, as then a type could be usefully used without appearing in |Rules|, and would not be missing from the table.

When deserialising, if we encounter a type not present in the table, we ignore the entire database. This is pessimistic, but safe -- if the set of types has changed that implies the build system has changed, which is not tracked anyway (see \S\ref{sec:changing_makefile}).

\subsection{Shake State}

\begin{figure}
\begin{code}
type State = Map Key Status

data Status  =  Loaded Info
             |  Building (MVar ()) (Maybe Info)
             |  Built Info

data  Info = Info
      {  value :: Value
      ,  built :: Time
      ,  changed :: Value
      ,  depends :: [[Key]]
      }
\end{code}

\xymatrix{
 & \mathsf{Loaded}\ar@@/_1pc/[dr]^{Invalid}\ar@@/^1pc/[rr]^{Valid} & & \mathsf{Built} \\
\mathsf{Database}\ar@@{-->}[ur] & & \mathsf{Building}\ar@@/_1pc/[ur]^{Finished} \\
& \mathsf{Missing}\ar@@{-->}[ur]
}
\caption{The state of a Shake process.}
\label{fig:state}
\end{figure}

The state of a running Shake program can be represented by the data type in Figure \ref{fig:state}. The |State| is a mapping from |Key| to |Status|, giving the status of each key the build system knows about. The |Status| data type has three constructors:

\begin{description}
\item[|Loaded|] is used for information that has been loaded from the database, but not yet checked against its history, or any stored value it may have.
\item[|Building|] is used for a rule that is currently running, or has been queued for execution. If a key is needed that is already building, the thread that requires it should wait on the |MVar| until that rule has completed.
\item[|Built|] is used for rules that were loaded and valid, or have finished building.
\end{description}

The |Status| constructors also take an |Info|, which contains the same fields as in \S\label{sec:unchanging_files}, with one modification. To allow full parallelism when rebuilding dependent rules, we store |depends| as |[[Key]]| instead of just |[Key]| -- where each entry comes from one call to |apply|. When saving |State| to the database, we ignore the |Status| fields and save only the |Key|/|Info| mapping.

With this state, it is relatively easy to implement the core of Shake -- the functions which are actually responsible for building account for less than 100 lines, most of which is dealing with parallelism. We execute the top-level |actions|, and whenever we call |apply|, we look up the state in the database, and use the transition diagram in Figure \ref{fig:state} to make sure everything is |Built| before continuing.

To implement parallelism we use the \prog{parallel-io} \cite{parallel-io} library to queue up rules to be run, which ensures we never have more than a user specified number of rules executing in parallel. The \prog{parallel-io} library deliberately randomises the order of computations it runs, as a result of the needs of Shake. By randomising the rules, we provoke failures earlier than we would other have. In addition, some stages of a build require different resources -- for example the compiler requires CPU while the linker requires disk access. If we are not careful, we can result in executing all compilers followed by all linkers. Using random ordering means we get a better mix, which gives a noticeable speedup -- measured at 20\% for some practical workloads.

When running a large build system, it is common for it to fail before completing -- either because a rule raised an error, or the end user killing the build process. In these cases it is important that none of the work that has been performed is lost -- therefore we maintain a journal and every time a rule transitions from |Building| to |Built| we write that information immediately. If Shake completes successfully we delete the journal. If on restarting Shake has a journal file, we replay the journal onto the database.


\subsection{Error Messages}

To improve error messages we make a number of changes from the natural implementation:

\begin{itemize}
\item We maintain a stack for each executing rule, containing the rules that required it. Whenever we invoke a rule, we check the stack for recursion in rules, giving a clear error message where otherwise there would have been an infinite loop.
\item Whenever an error occurs, either when running a rule or trying to find a rule, we print the stack. The stack is particularly useful when there is no rule for a file, but the fact that file is required is itself suspicious.
\item We have options to print every rule run, and every system command run. Whenever a system command fails, we reprint the command line after the failure.
\end{itemize}

\section{Evaluation}
\label{sec:evaluation}

We have used a build system based on Shake at Standard Chartered for the last three years. Our build system has been an unqualified success -- while the complexity of our system has increased, the build system has been able to handle all the change with incremental extensions. When introduced it replaced an extensive \make{} based system, which ran to over 10,000 lines of script, was regularly incorrect and very slow. Introducing Shake made our build system 10x shorter and 2x faster, even with the constraint of remaining backwards compatible with the old build system. Surveys have found that about 15\% of development work is expended on the build system \cite{make_survey}, we do far less.

In a single run our build system typically checks 360,000, although certain options will cause it to do even more. Our build system has up to 20 levels deep of |apply| calls from the top-level |action| statement, but 80\% of build rules are between 3 and 7 steps deep.

% [(0,2127),(1,22651),(2,190623),(3,64795),(4,15666),(5,22732),(6,10583),(7,11801),(8,5658),(9,5035),(10,3144),(11,2500),(12,974),(13,605),(14,251),(15,230),(16,73),(17,25),(18,25),(19,6),(20,1)]

In developing our build system we have learnt a number of lessons -- both about best practices for structuring build systems, and how the underlying core of Shake can be used to deal with the complexities of real development. In this section we share some of those lessons.

\subsection{Command Line Interface}

While the |shake| function is suitable for use as |main|, it offers no support for command line arguments by default. We use the |shake| function in a program \prog{mk}, which handles command line options, allowing us to specify options, such as parallelism and producing profiling reports.

One essential feature of any build tool is the @mk clean@ command, which deletes all build objects. We structure our build system to write all generated objects into a the folder \file{.make}, allowing us to implement @clean@ by deleting that folder. In a Shake system it is also possible to force a rebuild of all files by deleting the database. Finally, it is possible to clean a Shake build system by reading the database to determine which files were generated -- but we have found generating files alongside your source code to be unhelpful.

One feature of \make{} is that the targets can be specified on the command line. We allow a similar tactic, allowing both individual files and sets of files to be enabled/disabled. As an example, a user may write @mk !DOCS@ to disable building documentation, or @mk index.html@ to only build the target \file{index.html}. We control these targets by passing a modified version of |want| to the functions specifying rules, which consults the command line arguments:

\begin{code}
documentation :: (String -> [FilePath] -> Rules()) -> Rules ()
documentation wants = do
    wants "DOCS" ["index.html"]
    "index.html" *> \out -> ellipses
\end{code}

\subsection{Multiple Outputs}
\label{sec:multiple_outputs}

Some programs, such the Haskell compiler \prog{ghc} \cite{ghc}, can produce two outputs as a result of one action -- compiling \file{Foo.hs} produces both \file{Foo.o} and \file{Foo.hi}. As a first approximation, the \file{.o} file depends on the entire contents of the file, which the \file{.hi} file depends only on the type signatures. A single \prog{ghc} invocation needs to do all the work to do both, but often the \file{.hi} file will be left unchanged. Producing two files captures this pattern. Unfortunately, many build systems (\make{} included) do not handle multiple outputs well. In Shake, it is usually possible to fake a dependency -- claiming that \file{Foo.hi} depends on \file{Foo.o}, while \file{Foo.o} depends on \file{Foo.hs}, thanks to suitable support for unchanging files (see \S\ref{sec:unchanging_files}).

Unfortunately fake dependencies are not always enough. Consider a process that reads \file{numbers.txt} containing lines of numbers, and splits produces \file{even.txt} and \file{odd.txt} -- each containing only the even or odd numbers -- but does not update a file that has not changed. Here there is no fake dependency that can solve the problem. Fortunately, the power of multiple types of rules comes to the rescue. We define a new type, |File2|, which stores two files.

\begin{code}
data File2 = File2 FilePath FilePath
data FileTime2 = FileTime2 Int Int

rules = do
    rule $ \(File2 x y) -> if (x,y) /= ("even.txt","odd.txt") then Nothing else Just $ do
        need ["numbers.txt"]
        system' ["number-split","numbers.txt"]
        liftM2 File2 (getFileTime x) (getFileTime y)

    ["even.txt","odd.txt"] **> \_ ->
        apply [File2 "even.txt" "odd.txt"]
\end{code}



\subsection{Transitive Dependencies}

We can define the implementation as... this only does one layer.

We can also define a more specific one, based on the assumption that @#ifdef@ statements do not change the include files (a reasonable assumption in many cases).

If you have a transitive dependency you can avoid it quite easily. Consider the problem of .c/.h include files. We can define a correct implementation as...

\subsection{Build rules that change}

Throughout this work, we assume that the build script does not change, merely the inputs. We make some allowances and checks for this case, but generally if you change the build script you should wipe the build. We also in occasion depend on the file itself -- but that is only a crass generalisation, and is most useful when writing code generators.

We have actually moved most of our stuff into a build.mk file which is fully tracked, and would suggest others do the same. It should not be the build systems job to list which files go into a .c file, but perhaps have a list of those files somewhere in a text file.

\section{User Tools}
\label{sec:tools}

On top of Shake we can build a number of useful tools.

\subsection{Profiling}

In addition to the information listed in |Info|, we also record the duration of each rule, excluding it's children. With this we can predict how long a rule will take to execute, and which rules are most expensive. We also have a trace associated with each rule, along with a trace function:

\begin{code}
traced :: String -> IO a -> Action a
\end{code}

This code allows us to tag a particular action with timing information and a message. Here we record the actual start and end times. We automatically insert such commands for all system commands. Given that most build commands are system commands, we can now trace how many system commands were executing at each stage. For example we can show the following graph.

Storing the profile information is easy enough, but we also allow interactive filtering and reports with Javascript/HTML. We allow grouping of rules using the same wildcard syntax as in \ref{fig:file_rules}. Note that there is no way to group rules as per their definition with |rule|, since rules do not have a unique identifier or name -- only a higher order function to match them. We could require all rules to be given names, which would make profiling easier, but at the cost of complicating everything else. Given that profiling reports and viewed rarely, it seems the wrong trade off -- also people often want to slice in different ways. As an example, given a project with five subprojects, each of which consists of both C and Haskell, it is both interesting to see profiling information C vs Haskell, and also of the five projects against each other.

In previous versions of Shake we only recorded profiling information when explicitly asked, and only start/end times of system commands. We found that by unconditionally enabling profiling, and by including more information, we were able to more accurately find bottlenecks.

\subsection{Analysis}

Often, after running a compile, people ask ``Why did that file rebuild?''. Using the Shake analysis tools those questions can be answered. It naturally complements profiling, being that to improve the build you can speed up a rule, make it depend on less (so it rebuilds less often). Using the information stored by profiling we can also run what-if analysis, to figure out what the most expensive file in the project is. All the information is in Shake, but have found the following questions most likely:

\begin{itemize}
\item In my last execution, why was a particular file rebuilt? Shake can show the complete path of dependencies.
\item What is the most expensive file to rebuild? Shake scans every file, assuming that there are no savings due to unchanging files (see \S\ref{sec:unchanging_files}, and produces a report. Often, after seeing the top candidate, the user wishes to exclude that and continue, so Shake produces the files in order.
\item If I touch this file, what will rebuild? Shake provides similar answers to the previous questions, including a projected time.
\end{itemize}

Shake contains full information on dependencies, so could be used to answer more questions, but these seem to be the common ones.

\subsection{Lint checking}

There are two primary things we check for with lint. To get a full build we build everything, run whatever clean action the user has specified, then rerun each rule. The best lint will come if you:

* clean everything, apart from the database
* build with --lint

It's important when rebuilding to require the rules bottom up, i.e. the want statements should be replaced with something doing a bottom up traversal.

lint with invariants can always be done very easily.

lint with observations is much harder, since you don't want observations to overlap. that means you have to run the rules bottom up. or can try and subtract rules?

\subsubsection{Invariant information}

Things like whether a file exists or not should be constant throughout, otherwise any files which don't know will get confused. This can be checked by cleaning, and then running. Any key that marks itself as invariant should not change during this process. To do so, we add to the |Rule| class:

\begin{code}
invariant :: key -> Bool
invariant _ = False
\end{code}

\subsubsection{Dependency creation}

When running a file we can ask what has been changed, and what has been used. When we run just a rule, but no dependencies, the list of things that change must all be things who had rules run on them, and the list of things that were used should be equivalent.

\begin{code}
data Observed alpha = Observed {created :: Maybe [alpha], used :: Maybe [alpha]}

observed :: IO alpha -> IO (Observed key, alpha)
observed act = do res <- act; return (Observed Nothing Nothing, res)
\end{code}

Note that our original design simply passed an |IO ()| to observed, and required the function to execute it. Alas, this went wrong since it's too easy to write an incorrect version that fails to execute! Now, thanks to parametricity, if you don't pass an error back, you must have got it right!

Given a set of observations, we can check what a key used in its body, and what it declared to create. However, there are a number of situations where people don't write 100\% accurate dependencies, but the effect is equivalent. For example, assuming \file{temporary.txt} is unused in the result of the build system, the following two rules seem morally equivalent:

\begin{code}
"foo.txt" *> \out -> copyFile "bar.txt" out

"foo.txt" *> \_ -> need ["temporary.txt"]
"temporary.txt" *> \out -> do
    copyFile "bar.txt" "foo.txt"
    writeFileLines "temporary.txt" []
\end{code}

We can imagine more complex examples, and do define such examples in \ref{sec:multiple_outputs}. Therefore, we require the following properties:

\begin{itemize}
\item If you create a key that someone else requires, you must be a dependency of them.
\item If you use a key, you must depend on that key directly.
\item If you use a key but do not require it, you are safe, but conservative.
\end{itemize}

\section{Related Work}
\label{sec:related_work}

Build systems are an essential part of any large software project, yet often prove surprisingly tricky to get right. While most languages have a nice tool for building single language projects (ocamlbuild, ghc --make, Visual Studio projects), when building more complex multi-language projects, most people turn to \make{}. While there are many \make{} competitors (Scons, CMake, Jam, Ant, Waf), none have gained universal acceptance. We present Shake, a new build system based on a more powerful approach, which can do things \make{} cannot -- handling generated files properly. We have implemented Shake in Haskell, as a Haskell library, and it is used heavily -- compiling 10's of millions of lines of code per day.

A standard build system generates a dependency graph for all files, and processes them to meet the dependency constraints. While many have taken the approach of packaging dependency graph functionality in alternative ways, we here review a build systems which have a particular approach:

\subsection{Redo}

https://github.com/apenwarr/redo

Redo was originally described by djb, and then separately implemented later. It has several features in common with Shake, including the use of a central database to track dependencies. Focusing on the differences, each rule in a Redo script is a separate file. This reduces the abstraction, and complicates defining a build system, has a very limited rule matching format (just by extension), but has the upside of simplicity (there is no redo syntax, just shell script), and you can depend on the contents of a rule. You can only specify simple files like default.o, and it's hard to get them in the right place. The language of choice are shell scripts, although this can be altered with she-bang lines. The advantage of separate files is it can depend on a rule, which is cool. It doesn't have rules for anything other than files, which means it can't deal with multiple outputs nicely.

It can only track decisions that are files, which is annoying.

\subsection{Ninja}

Ninja the Chromium build system has dynamic dependencies (you can specify a rule that contains dependencies), multiple outputs. It focus on only running command lines, and is a bit more limited, but still encompasses many of the same ideas.

http://martine.github.com/ninja/manual.html

http://google-engtools.blogspot.com/2011/08/build-in-cloud-how-build-system-works.html

\subsection{Tup}

https://gittup.org/tup

\subsection{Haskell Build Libraries}

Haskell libraries: Blueprint, Coadjute, nemesis, cake, hake, hmk, zoom, openshake, abba

The |rule|/|apply| functions are checked at runtime. However, since everyone uses sugared versions, that isn't an issue. It should be relatively easy to define a statically typed rule that accumulates its types used, but it's more complex. OpenShake did it using type functions. If you thread the types through teh Rules type as a type argument you can also solve the serialisation of existentials.

\subsection{Comparison to @ghc --make@}

Compared to GHC --make, building the Shake project takes 9 seconds to compile, while GHC --make takes 6 seconds. The reason for the difference is that GHC has to analyse dependencies once, and can keep .hi information in memory. However, Shake can fight back with some parallelism - in the Shake demo, if we scale each compilation to 5seconds, then GHC would take at least 70 seconds, while Shake can gain with parallelism to take 45 seconds. Compared to something like gcc, which is always invoked in single mode, we do better.

However, we do far better for a null rebuild. We have essentially reduced it to a question of which files already exist, and which timestamps do a set of files -- to rebuild GHC must ask at least these questions, and in practice will look at the insides of each file. As a result, we can do a minimal rebuild in 0.03 seconds, while GHC takes 0.6 seconds. Our only unnecesary overhead is reading in the database.

\section{Conclusions}
\label{sec:conclusions}

Shake is awesome. Any future work goes here.

\subsection*{Acknowledgements}

Thanks to Standard Chartered, where the software was developed. Thanks to Raphael Montelatici for the name Shake. Thanks to Evan Laforge for discussions, particularly about multiple files.


\bibliographystyle{plainnat}
\balance
\bibliography

\end{document}
